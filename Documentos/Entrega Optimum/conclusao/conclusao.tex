A partir da análise previamente desenvolvida sobre os dados históricos da central de atendimentos, foi possível criar modelos de simulação que permitiram analisar diversos cenários de contratação para a operação da empresa no ano de 2022. A visualização da estrutura do negócio por meio de um dashboard facilitou a compreensão dos problemas da empresa e a formulação de propostas de solução para a gerência.\par
Propostas para a melhoria da central de atendimento, buscando cumprir a meta da gerência de atendimento de 90\% das ligações em até um minuto, incluem: estratégias de contratação, possibilidade de subcontratação, estratégias para redução dos tempos de atendimento (tempo de resolução dos problemas dos clientes), utilização de chatbots para demandas mais simples, implementação de uma metodologia de classificação de tipos de ligação mais efetiva e melhoria da comunicação com os clientes, visando fornecer um serviço de maior efetividade.\par
Todas as propostas apresentadas dependem das características do negócio e disponibilidade de recursos da gerência para desenvolvê-los. Devido à falta de informações não é possível, por exemplo, avaliar se a estrutura física atual da empresa comporta o trabalho de mais de 4 funcionários, o que pode implicar em custos extras para implementação dessa proposta.\par
Acreditamos que a consideração destas recomendações por parte da gerência da central de atendimentos é suficiente para trazer as operações de volta ao nível desejado.