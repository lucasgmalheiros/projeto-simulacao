\subsection{Descrição do problema}
\label{section: descricao}
O trabalho construído diz respeito à uma central de atendimentos que funciona das 8:00 às 18:00 (turno diário de dez horas) e apresenta quatro funcionários encarregados do processamento das ligações (atendimento).

O caso apresentado é relacionado ao desempenho da central de atendimentos. A meta fundamental é de atender 90\% das chamadas em até 1 minuto, e a central opera com seus atendentes disponíveis durante o horário comercial de segunda a sexta-feira.

O objetivo do trabalho proposto é analisar o desempenho da central de atendimento e fazer recomendações para que a empresa possa cumprir sua meta de desempenho. Para isso, serão utilizadas técnicas de simulação de eventos discretos e análise de dados históricos.

Algumas das análises esperadas pelo gerente incluem a determinação do volume máximo de chamadas que pode ser tratado enquanto ainda atingindo a meta de desempenho de 90\% em 1 minuto. Além disso, será construído um painel de controle para mostrar o desempenho da central em cada dia, com indicadores-chave de desempenho (KPIs) apresentados. Serão explorados diferentes níveis de agregação, como trimestral, mensal e semanal, para determinar qual é o mais relevante para a gestão da central de atendimento.

Por fim, será proposta uma série de recomendações para o gerente da central de atendimento, a fim de que ele possa tomar medidas para melhorar o desempenho da empresa e atingir novamente a meta estabelecida.

O sistema para a simulação discreta foi construído com base em um ACD (Activity cicle diagram) que analisa as interações ocorrentes entre as entidades participantes do processo. Posteriormente, com os dados estátiscos e as distribuições de chegadas das ligações conhecidas, foi possível realizar a simulação do evento utilizando o software ARENA, que deposita suas informações em um documento CSV. 

Com as informações a respeito da simulação, utilizando a integração com python e suas bibliotecas: pandas, seaborne, numpy e scipy.stats, foi possível conferir os resultados e validalos estatisticamente.

\subsection{Alterações da versão anterior}
\begin{itemize}
    \item Descrição mais detalhada do problema da central de atendimentos na Seção \ref*{section: descricao};
    \item Os gráficos de linhas das Figuras \ref*{fig: chamados-tempo}, \ref*{fig: t_servico-tempo}, \ref*{fig: espera-tempo} e \ref*{fig: arrivals-tempo} foram substituídos por gráficos de dispersão;
    \item A Figura \ref*{fig: correlogram} foi adicionada ao estudo de correlação das variáveis na Seção \ref*{section: correlacao-anual} para auxiliar na visualização do resultado obtido pelo teste de correlação de Spearman;
    \item Realização dos testes de aderência mensais para tempos entre chegadas na Seção \ref*{section: fit-arrivals};
    \item Escrita da Seção \ref*{section: simulacao}, que apresenta o estudo de simulação;
    \item As Seções \ref*{section: intro} (Introdução) e \ref*{section: conclusao} (Conclusão) foram reorganizadas para a escrita de novos conteúdos;
    \item Correção de erros gerais de redação e referências.
\end{itemize}