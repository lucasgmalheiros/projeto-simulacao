A partir da análise previamente desenvolvida sobre os dados históricos da central de atendimentos, foi possível modelar os tempos entre as chegadas das ligações e o tempo de atendimento dos funcionários, possibilitando a implementação de um modelo de simulação com auxílio do software Arena. O modelo foi descrito e testado com os dados históricos fornecidos, sendo posteriormente validado. Isso significa que, apesar do modelo não representar completamente a realidade, por se tratar de uma simplificação do processo, ainda assim ele é capaz de representar o comportamento geral dos dados reais.\\
A partir do modelo validado, realizou-se o teste de capacidade da central de atendimentos, que foi capaz de atender 229 chamadas por dia com 90\% delas sendo atendidas em até 1 minuto.\\
Direções futuras do trabalho incluem simulações para horizontes futuros de planejamento, recomendações para a central voltar a cumprir a meta após o crescimento no ano de 2021 e a construção de um painel de controle com indicadores de desempenho.