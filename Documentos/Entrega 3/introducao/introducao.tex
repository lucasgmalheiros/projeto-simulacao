\subsection{Descrição do problema}
\label{section: descricao}
O trabalho desenvolvido diz respeito a uma central de atendimentos que funciona das 8:00 às 18:00 (turno diário de dez horas) e apresenta quatro funcionários encarregados do processamento das ligações (atendimento).

O caso apresentado é relacionado ao desempenho da central de atendimentos. A meta fundamental é de atender 90\% das chamadas em até 1 minuto, e a central opera com seus atendentes disponíveis durante o horário comercial de segunda a sexta-feira.

O objetivo do trabalho proposto é analisar o desempenho da central de atendimento e fazer recomendações para que a empresa possa cumprir sua meta de desempenho. Para isso, serão utilizadas técnicas de simulação de eventos discretos e análise de dados históricos.

Algumas das análises esperadas pelo gerente incluem a determinação do volume máximo de chamadas que pode ser tratado sem violar a meta de desempenho de 90\% das chamadas atendidas em até 1 minuto. Além disso, será construído um painel de controle para mostrar o desempenho da central em cada dia, com indicadores-chave de desempenho (KPIs) apresentados. Serão explorados diferentes níveis de agregação, como trimestral, mensal e semanal, para determinar qual é o mais relevante para a gestão da central de atendimento.

Por fim, será proposta uma série de recomendações para o gerente da central de atendimento, a fim de que ele possa tomar medidas para melhorar o desempenho da empresa e atingir novamente a meta estabelecida.

O sistema para a simulação discreta foi construído com base em um ACD (Activity cicle diagram) que analisa as interações ocorrentes entre as entidades participantes do processo. Posteriormente, com os dados estatísticos e as distribuições de chegadas das ligações conhecidas, foi possível realizar a simulação do evento utilizando o software ARENA, que deposita suas informações em um documento CSV. 

Com as informações a respeito da simulação, e utilizando a linguagem Python para análise de dados, foi possível avaliar os resultados e validá-los estatisticamente.

\subsection{Alterações da versão anterior}
\begin{itemize}
    \item Apresentação, na Seção \ref*{section: sim-2021}, dos resultados obtidos pela execução do modelo para todo o ano de 2021;
    \item Na Seção \ref*{section: sim-2022} é apresentada a estratégia de simulação dos possíveis cenários de operação para o ano de 2022 da central de atendimentos;
    \item A Seção \ref*{section: propostas} descreve propostas de melhoria para que a central de atendimentos volte a cumprir a meta de 90\% dos atendimentos realizados em até 1 minuto;
    \item Reformulação da Seção \ref*{section: conclusao} (conclusão).
\end{itemize}