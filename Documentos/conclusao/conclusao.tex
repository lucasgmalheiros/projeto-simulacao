A análise de dados foi um importante processo para o projeto de simulação que está sendo desenvolvido. Através dela conseguimos compreender, e provar, quais são os melhores níveis de agregação, as distribuições que melhor representam os dados do projeto, o comportamento dos parâmetros e as tendências sob as quais estes se apoiam.

Restam como etapas a modelagem dos processos a serem simulados, bem como dos dados que estes receberão durante a simulação. Em especial, a dos tempos de chegada, que são não estacionários e, por consequência, são melhor representados por uma distribuição de Poisson do que por uma exponencial. Desta maneira, esse será um dos futuros esforçoes do projeto.

Outros desses esforçoes serão incluir o processo de tendência na simulação e na tomada de decisão para o cliente, garantindo uma assertividade maior ao projeto. Afinal, é esperado que o problema apresentado volte a ocorrer no tempo se a solução oferecida apenas resolva o estado presente do sistema.