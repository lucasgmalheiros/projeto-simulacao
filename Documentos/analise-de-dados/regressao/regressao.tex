\subsection{Estudos de Regressão}

Ao montarmos a visualização dos tempos de chegada ao longo do tempo percebemos que havia tendência nos dados. Para isso, primeiro, calculamos todos os tempos de chegada das chamadas, subtraindo o horário de recebimento de uma chamada subsequente pelo da anterior, isto é:

$$Tempo_{chegada}(c_n) = t_n - t_{n-1}$$ 


Depois, as chamadas foram classificadas por dia, e, assim, calculado o tempo de chegada médio diário delas. O principal intuito desta etapa é facilitar a visualização da evolução desse parâmetro ao longo do tempo. O gráfico obtido por essa operação está abaixo, na figura 13.

\begin{figure}[H]
    \includegraphics{analise-de-dados/regressao/tempo_chegada_medio.png}
    \caption{Tempo médio de chegada ao longo do Ano}
    \label{fig: tempos_de_chegada}
\end{figure}

A partir dessa percepção, são, então, propostos alguns estudos de regressão mais particulares para representar a tendência dos dados. Um usando "Ordinary Least Squares" e uma regressão exponencial.  

\subsection{Ordinary Least Squares}

Utilizando a biblioteca statsmodel do python foi realizado o estudo de regressão. A Equação Obtida foi: $$intervalos(t) = -0,4933 \cdot t + 252,6629$$. O resumo da regressão e o plot podem ser colocados a seguir: 

\begin{figure}[H]
    \includegraphics{analise-de-dados/regressao/regressao_OLS.png}
    \caption{Regressão Linear para os tempos de espera}
    \label{fig: plot_OLS}
\end{figure}

\begin{figure}[H]
    \includegraphics[scale = 0.85]{analise-de-dados/regressao/OLS_summary.jpg}
    \caption{Resumo da regressão}
    \label{fig: sum_OLS}
\end{figure}

Temos que os valores p são menores que o nível de significância adotado, e, portanto, o modelo de regressão existe. Entretanto, para os dias em que $t > 512$ o tempo de chegada das chamadas seria negativo, o que levanta questões sobre se este método de regressão é aplicável para modelar o tempo entre as chamadas.

\subsection{OLS exponencial}

De maneira semelhante ao primeiro estudo, este também foi feito utilizando a biblioteca statsmodel. Entretanto, ao invés dos parâmetros para a regressão terem sido adicionados como são ao modelo, ele foi construído usando: $$ln(y) = ln(b) + a \cdot x$$

Dessa maneira, a equação obtida para representar os dados foi:
$$intervalos(t) = 260,9424 \cdot e^{-0,0027x} $$

O resumo e o gráfico da regressão estão nas imagens 16 e 17:
\begin{figure}[H]
    \includegraphics{analise-de-dados/regressao/regressao_EXPO.png}
    \caption{Regressão Exponencial para os tempos de espera}
    \label{fig: plot_Expo_OLS}
\end{figure}

\begin{figure}[H]
    \includegraphics[scale = 0.85]{analise-de-dados/regressao/EXPO_OLS_Summary.jpg}
    \caption{Resumo da regressão}
    \label{fig: sum_Expo_OLS}
\end{figure}

Para Este caso, a explicabilidade do modelo aumentou e o problema que antes existia já não mais existe. Como os Valores P seguem menores do que 0, o modelo existe e pode ser usado.
