\subsection{Descrição do projeto}
Na primeira etapa do projeto, foi realizada a modelagem e analíse dos dados referentes a ligações recebidas por uma central de atendimentos.\\
Para isso, foram analísadas diferentes variáveis para que seu comportamento fosse análisado.\\
\begin{tabular}{rlrlllrrlr}
    \toprule
     call\_id &    date    &  daily\_caller &     call\_started    & call\_answered & call\_ended &  wait\_length &  service\_length &  meets\_standard &  call\_type \\
    \midrule
       1     & 2021-01-01 &       1       & 1900-01-01 08:00:00 &   8:00:00 AM  & 8:14:22 AM &      0       &       863       &       True      &     1      \\
       2     & 2021-01-01 &       2       & 1900-01-01 08:02:42 &   8:02:42 AM  & 8:07:31 AM &      0       &       289       &       True      &     0      \\
       3     & 2021-01-01 &       3       & 1900-01-01 08:08:24 &   8:08:24 AM  & 8:10:13 AM &      0       &       108       &       True      &     1      \\
       4     & 2021-01-01 &       4       & 1900-01-01 08:09:37 &   8:09:37 AM  & 8:13:45 AM &      0       &       247       &       True      &     1      \\
       5     & 2021-01-01 &       5       & 1900-01-01 08:11:10 &   8:11:10 AM  & 8:15:28 AM &      0       &       258       &       True      &     1      \\
    \bottomrule
    \end{tabular}
\\
Acima, pode-se observar 5 ligações registradas dentro da base de dados, bem como todas as variáveis. Call\_id é a identificação da ocorrência, date é a data, daily\_caller registra as ligações num dia, call\_started, call\_answered e call\_ended são os momentos de ligação, atendimento e término das chamadas. 
wait\_length é o tempo de espera,  service\_length é o tempo de atendimento, meets\_standard retorna verdadeiro caso o tempo de espera seja menor do que um minuto e call\_type é o tipo da ligação.\\
Para os testes de hipóteses atrelados às diferenças entre essas variáveis em diferentes períodos foi utilizado o método Kolmogorov-Smirnov para duas amostras que possui caráter não parametrico, sendo assim:\\
\begin{center}
$H{0}$ : Amostras seguem a mesma distribuição\\
$H{a}$ : Amostras seguem distribuições diferentes
\end{center}
Posteriormente a confirmação positiva da diferença entre amostras, foi realizado o teste de aderência das amostras às distribuições cauchy, chi quadrado, exponencial, gamma, lognormal, normal, powerlaw, rayleigh, uniforme.\\
Por último para analisar as correlações entre variáveis, foram realizadas regressões lineares e exponeciais.\\

\subsection{Próxima etapa}
Para a próxima etapa do processo, será realizada a modelagem da diferença entre chegadas (ligações) no sistema, para que posteriormente seja desenvolvida a simulação no ARENA.