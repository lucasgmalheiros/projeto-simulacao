\subsection{Descrição do problema}
\label{section: descricao}
O trabalho desenvolvido diz respeito ao estudo de caso da operação de uma central de atendimentos proposto como projeto final da disciplina de simulação de sistemas oferecida pelo Departamento de Engenharia de Produção da Universidade Federal de São Carlos. Esta central funciona das 8:00 às 18:00 (turno diário de dez horas) e possui quatro funcionários, todos encarregados do atendimento das ligações, que estão disponíveis durante o todo o horário de funcionamento de segunda a sexta-feira.

O objetivo do trabalho é analisar o desempenho da central de atendimentos e fazer recomendações para que a empresa possa cumprir sua principal meta estabelecida pela gerência, que é a de atender 90\% das chamadas em até 1 minuto. Para isso, serão utilizadas técnicas de simulação de eventos discretos e análise de dados históricos.

Algumas das análises esperadas pelo gerente incluem: determinação do volume máximo de chamadas que pode ser tratado sem violar a meta de desempenho de 90\% das chamadas atendidas em até 1 minuto, construção de painel de controle para mostrar o desempenho da central em cada dia, com indicadores-chave de desempenho (KPIs) apresentados, e exploração de diferentes níveis de agregação, como trimestral, mensal e semanal, para determinar qual é o mais relevante para a gestão da central de atendimentos.

Por fim, será proposta uma série de recomendações para o gerente da central de atendimentos, a fim de que ele possa tomar medidas para melhorar o desempenho da empresa e atingir novamente a meta estabelecida.

O sistema para a simulação discreta foi construído com base em um ACD (Activity cicle diagram) que analisa as interações entre as entidades participantes do processo. Posteriormente, com os dados estatísticos e as distribuições de chegadas das ligações conhecidas, foi possível realizar a simulação do evento utilizando o software ARENA, que deposita suas informações em um documento CSV. 

Com as informações a respeito da simulação, e utilizando a linguagem Python para análise de dados, foi possível avaliar os resultados e validá-los estatisticamente.
