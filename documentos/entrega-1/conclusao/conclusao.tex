A análise de dados foi um importante processo para o projeto de simulação que está sendo desenvolvido. Por meio dela conseguimos compreender, e testar, quais os melhores níveis de agregação, as distribuições que melhor representam os dados do projeto, o comportamento dos parâmetros e as tendências sob as quais estes se apoiam. Dessa forma, foi possível definir o nível de agregação mensal para o planejamento das atividades do call center, por ser o nível que melhor representa o crescimento do número de chegadas ao longo do ano, assim como modelar os tempos de serviço para o ano inteiro.

Restam como etapas a modelagem dos processos a serem simulados, bem como dos dados que estes receberão durante a simulação. Em especial, a dos tempos entre chegadas, que são não estacionários e, por consequência, são melhor representados por um processo de Poisson ou algum outro método que leve em consideração o tempo decorrido na simulação. Desta maneira, esse será um dos futuros esforços do projeto.

Outros desses esforços serão incluir o processo de tendência na simulação e na tomada de decisão para o cliente, garantindo uma assertividade maior ao projeto. Afinal, é esperado que o problema apresentado volte a ocorrer no tempo se a solução oferecida apenas resolva o estado presente do sistema.